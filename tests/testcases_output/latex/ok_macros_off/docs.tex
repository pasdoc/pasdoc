\documentclass{report}
\usepackage{hyperref}
% WARNING: THIS SHOULD BE MODIFIED DEPENDING ON THE LETTER/A4 SIZE
\oddsidemargin 0cm
\evensidemargin 0cm
\marginparsep 0cm
\marginparwidth 0cm
\parindent 0cm
\setlength{\textwidth}{\paperwidth}
\addtolength{\textwidth}{-2in}


% Conditional define to determine if pdf output is used
\newif\ifpdf
\ifx\pdfoutput\undefined
\pdffalse
\else
\pdfoutput=1
\pdftrue
\fi

\ifpdf
  \usepackage[pdftex]{graphicx}
\else
  \usepackage[dvips]{graphicx}
\fi

% Write Document information for pdflatex/pdftex
\ifpdf
\pdfinfo{
 /Author     (Pasdoc)
 /Title      ()
}
\fi


\begin{document}
\label{toc}\tableofcontents
\newpage
% special variable used for calculating some widths.
\newlength{\tmplength}
\chapter{Unit ok{\_}macros{\_}off}
\label{ok_macros_off}
\index{ok{\_}macros{\_}off}
\section{Description}
This is a test that macro support can be turned off in pasdoc.\hfill\vspace*{1ex}



With macro support turned on this unit would cause parsing error, because it would have brain{-}damaged declaration like \texttt{\\\nopagebreak[3]
}\textbf{procedure}\texttt{~}\textbf{interface}\texttt{~}\textbf{interface}\texttt{(a:~Integer);\\
}
\section{Overview}
\begin{description}
\item[\texttt{FOO}]
\end{description}
\section{Functions and Procedures}
\ifpdf
\subsection*{\large{\textbf{FOO}}\normalsize\hspace{1ex}\hrulefill}
\else
\subsection*{FOO}
\fi
\label{ok_macros_off-FOO}
\index{FOO}
\begin{list}{}{
\settowidth{\tmplength}{\textbf{Description}}
\setlength{\itemindent}{0cm}
\setlength{\listparindent}{0cm}
\setlength{\leftmargin}{\evensidemargin}
\addtolength{\leftmargin}{\tmplength}
\settowidth{\labelsep}{X}
\addtolength{\leftmargin}{\labelsep}
\setlength{\labelwidth}{\tmplength}
}
\item[\textbf{Declaration}\hfill]
\ifpdf
\begin{flushleft}
\fi
\begin{ttfamily}
procedure FOO(a: Integer);\end{ttfamily}

\ifpdf
\end{flushleft}
\fi

\end{list}
\end{document}
